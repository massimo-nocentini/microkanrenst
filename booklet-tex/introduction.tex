
This document extends the presentation of $\mu$Kanren shown at the last ESUG
conference \citep{ESUG2018} in two ways: (i)~it recaps existing literature 
on the subject and (ii)~it describes in more details our implementation coded
in Smalltalk.

It is a waste of time to repeat ourself on $\mu$Kanren's definition; on the
contrary, we believe interesting to look up and to watch over the many
implementations it has inspired since its seminal paper \citep{Hemann:muKanren}
and how programmers stretch the canonical implementation to take advantage of
their favorite programming language's features.

Certainly the book \citep{Friedman:Reasoned:Schemer} is the main reference
written in the same style of the other \emph{little books} of which Dan
Friedman coauthors them all; moreover, Byrd's dissertation \citep{Byrd:PhD}
shows advanced stuffs and techniques while the repository
\url{http://minikanren.org/} -- updated regularly by Byrd himself -- keeps
pointers to implementations, articles and presentations.  $\mu$Kanren is
\emph{purely functional} therefore any language that has the same power of the
$\lambda$-calculus is a candidate for implementing it, nonetheless its first
definition has been written in Scheme; in parallel, the implementation with
symbolic constraints has the same nature but uses macros and in some aspect it
requires more effort to port it to other languages that haven't macros.

Another implementation using the \emph{Clojure} dialect of Lisp is
\citep{clojure:core.logic} which targets the JVM and provides constraints and
nominal logic programming; on the other hand, \citep{sullivan:microkanrenhs}
is a clean and concise implementation using Haskell, while many others simpler
versions can be found, written for the sake of understanding and, ultimately,
for fun.

Finally, your humble authors believe that $\mu$Kanren is a sophisticated
\textit{educational machine} because it concerns and mixes together unification,
lazy streams, backtracking and optimization techniques; moreover, it allows
programmers to write their own version according to their personal taste and
style, ranging from implementations pretty imperative to ones built over
categories. 

Our effort to understand $\mu$Kanren started in \citep{Nocentini:kanrens} using
the \emph{Python} programming language; we take advantage of native
\Verb|generator|s objects and we provide a \emph{fair} enumeration strategy
that the original implementation hasn't; on the contrary, we experience
computational limits because of \texttt{RecursionError} exceptions. This
implementation is complete and parallels the one shown in the reference book,
accompanied by a unit test suite that provides answers to \emph{all} questions
proposed in the book and some new applications such as (i)~finding fixed points
of a set of inference rules and (ii)~enumerating classes of combinatorial
objects. Moreover, two \textit{impure} operators, as the \textit{cut}
(\Verb|!|) Prolog operator, had been coded too. However complete, the code is a
mixture of the object orientation at the surface and functional at the core both
supported by primitives of the underlying language, such as \Verb|yield|
expressions and magic methods; took note of this, we desire something more
uniform and cohesive.

The current work describes a Smalltalk implementation of the basic system with
the aim to learn and experiment with a pure object environment, such as Pharo
and Squeak. The work \citep{evan:smallkanren} is another Smalltalk
implementation of $\mu$Kanren that currently supports disequality constraints
and nominal logic; it is a more complete and complex version that we point it
out for users requiring those features -- hoping to be able to provide those
features by ourselves.

Starting small, the next example shows three different
approaches to define the addition of three \ct{SmallInteger} objects toward 
the introduction of the relational paradigm.
\begin{example}
In math a relation $P$ is usually characterized by a statement such as
\begin{displaymath}
\forall a,b,c.\,P(a,b,c) \leftrightarrow a + b = c \quad\text{from which}\quad
P(1,2,3) \quad\text{can be proved.}
\end{displaymath}
The same check can be expressed in Smalltalk using either the
\textit{imperative style}
\begin{minted}[fontsize=\footnotesize]{smalltalk}
a := 1.  b := 2.  c := a + b.  Object assert: [ c = 3 ]
\end{minted}
or the \textit{functional style}
\begin{minted}[fontsize=\footnotesize]{smalltalk}
Object assert: [ ([ :a :b | a + b ] value: 1 value: 2) = 3 ]
\end{minted}
or, finally, the \textit{declarative style}
\begin{minted}[fontsize=\footnotesize]{smalltalk}
Object assert: [ [ :a :b :c | a + b = c ] value: 1 value: 2 value: 3 ]
\end{minted}
which closely resembles the mathematical statement.
\end{example}
