
\documentclass{beamer}

\usepackage{euler}
\usepackage[utf8]{inputenc}
\usepackage[T1]{fontenc}
\usepackage{textcomp}
\usepackage{minted}
\usepackage[scaled=0.8]{beramono}

\usepackage{amsmath}
\usepackage{amsthm}
\usepackage{amssymb}

\usefonttheme[onlymath]{serif}

\usemintedstyle{friendly}

\setbeamertemplate{blocks}[rounded]

%Information to be included in the title page:
\title{Relational Programming in Smalltalk}
\author{Massimo Nocentini}
\institute{University of Florence, Italy}
\date{ESUG2018}

\begin{document}

\frame{\titlepage}

\begin{frame}[fragile]
\frametitle{}
\begin{minted}[fontsize=\small]{smalltalk}
outline
    ^ LinkedList new
        add: 'Resolution by refutation';
        add: 'Resolution by unification';
        yourself
\end{minted}
\end{frame}

\begin{frame}[fragile]
\frametitle{Main idea}
\begin{displaymath}
\forall a,b,c.\,P(a,b,c) \leftrightarrow a + b = c \quad\text{entails}\quad P(1,2,3)
\end{displaymath}
can be expressed using either the \textit{imperative style}
\begin{minted}[fontsize=\footnotesize]{smalltalk}
a := 1.
b := 2.
c := a + b.
Object assert: [ c = 3 ].
\end{minted}
or the \textit{functional style}
\begin{minted}[fontsize=\footnotesize]{smalltalk}
([ :a :b | a + b ] value: 1 value: 2) = 3
\end{minted}
or, finally, the \textit{declarative style}
\begin{minted}[fontsize=\footnotesize]{smalltalk}
[ :a :b :c | a + b = c ] value: 1 value: 2 value: 3
\end{minted}

\end{frame}

\begin{frame}[fragile]

\frametitle{Resolution by \textit{Refutation}}

Let $\alpha$ be a sentence in \textit{CNF} and $M(\alpha)$ the set of models that
satisfy it, where a model is a set of assignments that make $\alpha$ true.
\vfill
$\alpha$ is \textit{valid} if it is true in \textit{all} models; oth,\\
$\alpha$ is \textit{satisfiable} if it is true in \textit{some} model.
\vfill
\textit{Logical reasoning} boils down to decide the \textit{entailment}
relation $\models$
\begin{displaymath}
\begin{split}
M(\alpha) &\subseteq M(\beta) \leftrightarrow \\
\alpha &\models \beta \leftrightarrow \\
(\alpha &\Rightarrow \beta) \text{ is valid } \leftrightarrow\\
\neg(\neg\alpha &\vee \beta) \text{ is unsatisfiable}
\end{split}
\end{displaymath}
therefore, to prove a sentence $\alpha$ reduces to decide
$\neg\alpha\models\perp$,\\where $\perp$ is the (\textit{unsatisfiable}) empty
clause.
\end{frame}

\begin{frame}[fragile]
\frametitle{Resolution by \textit{Refutation}}
The \textit{resolution rule} is a \textit{complete} inference algorithm; \\
let $l_{q}, m_{w} \in\lbrace 0,1\rbrace$ in
\begin{displaymath}
{\left(l_{0}, \ldots, l_{i}, \ldots, l_{j-1}\right) \quad \left(m_{0}, \ldots, m_{r},\ldots, m_{k-1}\right) \quad l_{i} = \neg m_{r}
\over
\left(l_{0},\ldots, l_{i-1},l_{i+1}, \ldots,l_{j-1}, m_{0},\ldots, m_{r-1},m_{r+1},\ldots, m_{k-1}\right)}
\end{displaymath}
where $\left(l_{0},\ldots, l_{i}, \ldots, l_{j-1}\right) = l_{0}\vee \ldots \vee l_{i} \vee \ldots \vee l_{j-1}$.
\vfill
The \textit{DPLL} algorithm is a recursive, depth-first enumeration of models
using the resolution rule, paired with heuristics \textit{early termination},
\textit{pure symbol} and \textit{unit clause} to speed up.
\end{frame}

\begin{frame}[fragile]
\frametitle{Resolution by \textit{Unification}}

\textit{Unification} is the process of solving \textit{equations among symbolic
expressions}; a \textit{solution} is denoted as a \textit{substitution}
$\theta$, namely a mapping that assigns a symbolic values to free variables.
\vfill
Let $x$ and $y$ be free variables, the set
$$\lbrace cons(x,cons(x,nil)) = cons(2,y)\rbrace$$
has solution $\theta = \lbrace x \mapsto 2, y \mapsto cons(2,nil) \rbrace$;
moreover, the set
$$ \lbrace y = cons(2,y) \rbrace $$
has no \textit{finite} solution; oth, $\theta = \lbrace y \mapsto
cons(2,cons(2,cons(2,...))) \rbrace$ is a solution by \textit{bisimulation}.
\end{frame}

\begin{frame}[fragile]
\frametitle{Resolution by \textit{Unification}}
let $G$ be a set of equations, unification rules are
\begin{description}
\item[delete] $G \cup \lbrace t = t \rbrace \rightarrow G$
\item[decompose] $G \cup \lbrace f(s_{0}, \ldots, s_{k}) = f(t_{0}, \ldots, t_{k})\rbrace$ entails
$$G \cup \lbrace s_{0}=t_{0},\ldots, s_{k}=t_{k} \rbrace$$
\item[conflict] if $f\neq g \vee k\neq m$ then $$G \cup \lbrace f(s_{0}, \ldots, s_{k}) = g(t_{0}, \ldots, t_{m})\rbrace \rightarrow \,\perp$$
\item[eliminate] if $x \not\in vars(t)$ and $x \in vars(G)$ then $$G \cup \lbrace x = t\rbrace \rightarrow G\lbrace x \mapsto t\rbrace \cup \left\lbrace x \triangleq t\right\rbrace $$
\item[occur check] if $x \in vars(f(s_{0},\ldots,s_{k}))$ then $$G \cup \lbrace x = t(s_{0}, \ldots, s_{k})\rbrace \rightarrow \,\perp$$
\end{description}
%without *occur checks*, generating a substitution $\theta$ is a *recursive enumerable* problem
\end{frame}

\begin{frame}[fragile]
\frametitle{microkanren}
\begin{itemize}
\item a \textit{DSL} for relational programming written in Scheme
\item \textit{purely functional} implementation of \textit{miniKanren}
\item \textit{explicit streams} of satisfying states, \textit{goal-based} approach
\item resolution by unification via \textit{structural induction}
\item \textit{complete}, but \textit{unfair}, search strategy
\end{itemize}

A \textit{goal} is an object that responds to the \verb|#onState:| selector, it
receives a \textit{substitution} and returns a \verb|Chain| object of
substitutions. 

\iffalse
my contribution
- _Pythonic_ [implementation][mkpy]: functional at the core, objective at the interface
- generators subsume _countably_-satisfiable relations; complete, _fair_ [search][dovetail]
- _The Reasoned Schemer_ fully tested via [Travis CI][travis]; moreover, [read the docs][rtfd]
- case studies: Smullyan puzzles and combinatorics
- tweaking HOL Light for _certified deductions_, [wip][klight]
\fi
\end{frame}

\begin{frame}[fragile]
\frametitle{Chain subclass: \#Bottom}

\begin{minted}[fontsize=\footnotesize]{smalltalk}
BlockClosure>>links: anObj
    ^ Chain item: anObj linker: self
\end{minted}
\begin{minted}[fontsize=\footnotesize]{smalltalk}
Chain class>>bottom
    ^ Bottom new
\end{minted}
\begin{minted}[fontsize=\footnotesize]{smalltalk}
Chain class>>item: anObj linker: aBlockClosure
    ^ Knot new 
        item: anObj;
        linker: aBlockClosure;
        yourself
\end{minted}
\begin{minted}[fontsize=\footnotesize]{smalltalk}
bind: aGoal interleaved: anInterleaved
    ^ self
\end{minted}
\begin{minted}[fontsize=\footnotesize]{smalltalk}
mplus: anotherChain interleaved: anInterleaved
    ^ anotherChain value
\end{minted}
\begin{minted}[fontsize=\footnotesize]{smalltalk}
atMost: anInteger
    ^ self
\end{minted}
\begin{minted}[fontsize=\footnotesize]{smalltalk}
mature
    ^ LinkedList new
\end{minted}
\end{frame}

\begin{frame}[fragile]
\frametitle{Chain subclass: \#Knot}
\begin{minted}[fontsize=\footnotesize]{smalltalk}
bind: aGoal interleaved: anInterleaved
    | alpha beta |
    alpha := aGoal onState: item.
    beta := [ self next bind: aGoal interleaved: anInterleaved ].
    ^ alpha mplus: beta interleaved: anInterleaved
\end{minted}
\begin{minted}[fontsize=\footnotesize]{smalltalk}
mplus: anotherChain interleaved: anInterleaved
    ^ [ :_ | anotherChain value 
                mplus: [ self next ] 
                interleaved: anInterleaved ] links: item
\end{minted}
\begin{minted}[fontsize=\footnotesize]{smalltalk}
next
    ^ linker value: item
\end{minted}
\begin{minted}[fontsize=\footnotesize]{smalltalk}
atMost: n
    ^ n isZero
        ifTrue: [ Chain bottom ]
        ifFalse: [ [ :_ | self next value atMost: n - 1 ] links: item ]
\end{minted}
\begin{minted}[fontsize=\footnotesize]{smalltalk}
mature
    ^ self next mature
        addFirst: item;
        yourself
\end{minted}
\end{frame}

\begin{frame}[fragile]
\frametitle{ChainTest}
\begin{minted}[fontsize=\footnotesize]{smalltalk}
ints: i
    ^ [ :a | self ints: a + 1 ] links: i
\end{minted}
\begin{minted}[fontsize=\footnotesize]{smalltalk}
fib: m fib: n
    ^ [ :_ | self fib: n fib: m + n ] links: m
\end{minted}
\begin{minted}[fontsize=\footnotesize]{smalltalk}
collatz: o
    ^ [ :_ | o even
                ifTrue: [ self collatz: o / 2 ]
                ifFalse: [ self collatz: 3 * o + 1 ] ] links: o
\end{minted}
\begin{minted}[fontsize=\footnotesize]{smalltalk}
testNumbers
    self assert: (self nats atMost: 10) mature equals: (0 to: 9).
    self
        assert: (self fibs atMost: 10) mature
        equals: {0 . 1 . 1 . 2 . 3 . 5 . 8 . 13 . 21 . 34}.
    self
        assert: ((self collatz: 10) atMost: 10) mature
        equals: {10 . 5 . 16 . 8 . 4 . 2 . 1 . 4 . 2 . 1}.
    self
        assert: ((self gcd: 18 gcd: 32) atMost: 10) mature
        equals: {18 . 32 . 18 . 14 . 4 . 2 . 2}
\end{minted}
\end{frame}

\begin{frame}[fragile]
\frametitle{Chain combinators}
\begin{minted}[fontsize=\footnotesize]{smalltalk}
Bind class>>combine: aGoal with: aCollection
    ^ self new
        combiner: aGoal;
        stream: aCollection;
        yourself
\end{minted}
\begin{minted}[fontsize=\footnotesize]{smalltalk}
Bind>>interleavedStrategy: anInterleaved
    ^ stream bind: combiner interleaved: anInterleaved
\end{minted}
\begin{minted}[fontsize=\footnotesize]{smalltalk}
MPlus class>>with: aCollection with: anotherCollection
    ^ self new
        left: aCollection;
        right: anotherCollection;
        yourself
\end{minted}
\begin{minted}[fontsize=\footnotesize]{smalltalk}
MPlus>>interleavedStrategy: anInterleaved
    ^ left mplus: right interleaved: anInterleaved
\end{minted}
\begin{minted}[fontsize=\footnotesize]{smalltalk}
Sequential>>of: aStreamCombination
    ^ aStreamCombination sequentialStrategy: self
\end{minted}
\begin{minted}[fontsize=\footnotesize]{smalltalk}
Interleaved>>of: aStreamCombination
    ^ aStreamCombination interleavedStrategy: self
\end{minted}
\end{frame}

\begin{frame}[fragile]
\frametitle{Goals}
\begin{minted}[fontsize=\footnotesize]{smalltalk}
Succeed>>onState: aState
    ^ Chain with: aState
\end{minted}
\begin{minted}[fontsize=\footnotesize]{smalltalk}
Fail>>onState: aState
    ^ Chain bottom
\end{minted}
\begin{minted}[fontsize=\footnotesize]{smalltalk}
State>>at: aVar put: aValue
    | s |
    s := substitution copy.
    s
        at: aVar
        ifPresent: [ :v | 
            aValue = v
                ifFalse: [ UnificationError signal ] ]
        ifAbsent: [ s at: aVar put: aValue ].
    ^ self class new
        birthdate: birthdate;
        substitution: s;
        yourself
\end{minted}
\end{frame}

\begin{frame}[fragile]
\frametitle{Goals}
\begin{minted}[fontsize=\footnotesize]{smalltalk}
Fresh>>onState: aState
    ^ aState collectVars: (1 to: receiver numArgs) forFresh: self
\end{minted}
\begin{minted}[fontsize=\footnotesize]{smalltalk}
State>>collectVars: aCollection forFresh: aFresh
    | nextState vars |
    nextState := self class new
        substitution: substitution;
        birthdate: birthdate + aCollection size;
        yourself.
    vars := aCollection collect: [ :i | Var id: i ].
    ^ aFresh onState: nextState withVars: vars
\end{minted}
\begin{minted}[fontsize=\footnotesize]{smalltalk}
Fresh>>onState: aState withVars: aCollection
    | g |
    vars := aCollection.
    g := receiver valueWithArguments: vars.
    ^ g onState: aState
\end{minted}
\begin{minted}[fontsize=\footnotesize]{smalltalk}
BlockClosure>>fresh
    ^ Goal fresh: self
\end{minted}
\end{frame}

\begin{frame}[fragile]
\frametitle{Goals}
\begin{minted}[fontsize=\footnotesize]{smalltalk}
Object>>unifyWith: another
    ^ Goal unify: self with: another
\end{minted}
\begin{minted}[fontsize=\footnotesize]{smalltalk}
Unify>>onState: aState
    ^ [ | extended_state |
        extended_state := Unifier new 
                            unify: this with: that onState: aState.
        Goal succeed onState: extended_state ]
            on: UnificationError
            do: [ Goal fail onState: aState ]
\end{minted}
\begin{minted}[fontsize=\footnotesize]{smalltalk}
Unifier>>unify: anObj with: anotherObj onState: aState
    | aWalkedObj anotherWalkedObj |
    aWalkedObj := aState walk: anObj.
    anotherWalkedObj := aState walk: anotherObj.
    ^ aWalkedObj unifyWith: anotherWalkedObj 
                 usingUnifier: self 
                 onState: aState
\end{minted}
\begin{minted}[fontsize=\footnotesize]{smalltalk}
State>>walk: anObj
    | k |
    k := anObj.
    [ k := substitution at: k ifAbsent: [ ^ k ] ] repeat
\end{minted}
\end{frame}

\begin{frame}[fragile]
\frametitle{Unifier}
\begin{minted}[fontsize=\footnotesize]{smalltalk}
unifyObject: anObj withObject: anotherObj onState: aState
    ^ anObj = anotherObj
        ifTrue: [ aState ]
        ifFalse: [ UnificationError signal ]
\end{minted}
\begin{minted}[fontsize=\footnotesize]{smalltalk}
unifyVar: aVar withObject: anObject onState: aState
    ^ aState at: aVar put: anObject
\end{minted}
\begin{minted}[fontsize=\footnotesize]{smalltalk}
unifyVar: aVar withVar: anotherVar onState: aState
    ^ aVar = anotherVar
        ifTrue: [ aState ]
        ifFalse: [ 
            self unifyVar: aVar withObject: anotherVar onState: aState ]
\end{minted}
\begin{minted}[fontsize=\footnotesize]{smalltalk}
unifyLinkedList: aCollection withLinkedList: anotherCollection onState: aState
    ^ aCollection size = anotherCollection size
        ifTrue: [ (aCollection zip: anotherCollection)
            inject: aState
            into: [ :s :pair | self unify: pair key 
                                    with: pair value 
                                    onState: s ] ]
        ifFalse: [ UnificationError signal ]
\end{minted}
\end{frame}


\begin{frame}[fragile]
\frametitle{Unifier}
\begin{minted}[fontsize=\footnotesize]{smalltalk}
Disj>>onState: aState
    ^ interleaving of: ((either onState: aState) 
        mplus: [ or onState: aState ])
\end{minted}
\begin{minted}[fontsize=\footnotesize]{smalltalk}
Conj>>onState: aState
    ^ interleaving of: ((both onState: aState) bind: and)
\end{minted}
\begin{minted}[fontsize=\footnotesize]{smalltalk}
Cond>>if: ifGoal then: thenGoal
    clauses add: ifGoal -> thenGoal
\end{minted}
\begin{minted}[fontsize=\footnotesize]{smalltalk}
Cond>>ifPure: aStrategy
    if := [ :clause :otherwise | 
            IfPure new
                question: clause key 
                    answer: clause value 
                    otherwise: otherwise;
                streamCombinationStrategy: aStrategy;
                yourself ]
\end{minted}
\end{frame}

\begin{frame}[fragile]
\frametitle{Unifier}
\begin{minted}[fontsize=\footnotesize]{smalltalk}
Cond>>e
    self ifPure: Sequential new
\end{minted}
\begin{minted}[fontsize=\footnotesize]{smalltalk}
Cond>>i
    self ifPure: Interleaved new
\end{minted}
\begin{minted}[fontsize=\footnotesize]{smalltalk}
Cond>>onState: aState
    | g |
    else ifNil: [ self else: false asGoal ].
    g := clauses copy
            add: else;
            reduceRight: if.
    ^ g onState: aState
\end{minted}
\end{frame}

\begin{frame}[fragile]
\frametitle{Unifier}
Let $\mathcal{D}$ be the set of \textit{Dyck paths} and let $\leadsto$ be the \textit{CFG}
$$
\leadsto = \varepsilon \,\left|\, ( \leadsto ) \leadsto \right.
$$
where $\varepsilon$ is the empty string; question: enumerate $\mathcal{D}$ using $\leadsto$.
\begin{minted}[fontsize=\footnotesize]{smalltalk}
dycko: alpha
    ^ Goal cond e
        if: alpha nilo then: true asGoal;
        else: [ :beta :gamma | 
            (sexpTheory let: alpha 
                        be: ($( cons: beta) 
                        append: ($) cons: gamma)) & 
            ([ self dycko: beta ] eta & 
             [ self dycko: gamma ] eta) ] fresh
\end{minted}
\end{frame}

\begin{frame}[fragile]
\frametitle{Unifier}
\begin{minted}[fontsize=\footnotesize]{smalltalk}
testDycko
    | g |

    "enumeration"
    g := [ :alpha | combTheory dycko: alpha ] fresh.
    self
        assert: (g solutions atMost: 20)
        equals:
            ({nil . '()' . '(())' . '()()' . '(()())' . '()(())' . 
            '(())()' . '()()()' . '(()()())'.  '()(()())' . 
            '(())(())' . '()()(())' . '((()))' . '()(())()' . 
            '(())()()' . '()()()()'.  '(()()()())' . '()(()()())' . 
            '(())(()())' . '()()(()())'} collect: #asCons).

    "an invalid Dyck path"
    g := [ :alpha | combTheory dycko: '(()(())()(' asCons ] fresh.  
    self assert: g solutions all equals: {}
\end{minted}
\end{frame}

\begin{frame}[fragile]
\frametitle{Unifier}
Let $X$ and $Y$ be natural numbers in machine 
\begin{displaymath}
\mathcal{C} =  \left \lbrace{ \over 2X \stackrel{\circ}{\rightarrow} X} , {X \stackrel{\circ}{\rightarrow} Y \over 3X \stackrel{\circ}{\rightarrow} Y 2 Y} \right \rbrace
\end{displaymath}
question: does exist a number $\alpha$ such that $ \alpha \stackrel{\circ}{\rightarrow} \alpha $?
\begin{minted}[fontsize=\footnotesize]{smalltalk}
consumes: two_alpha produces: alpha machine: aMachine
    ^ two_alpha unifyWith: (2 cons: alpha)
\end{minted}
\begin{minted}[fontsize=\footnotesize]{smalltalk}
consumes: three_alpha produces: alpha_two_alpha machine: aMachine
    ^ [ :beta :gamma | 
        (three_alpha unifyWith: (3 cons: beta)) &
        ((self associate: gamma is: alpha_two_alpha machine: aMachine) & 
        (aMachine proves: beta relates: gamma)) ] fresh
\end{minted}
\end{frame}

\begin{frame}[fragile]
\frametitle{Unifier}
\begin{minted}[fontsize=\footnotesize]{smalltalk}
InductiveRelationsTheory>>proves: anObj relates: anotherObj
    | g |
    g := Goal cond i.
    rules do: [ :r | g if: (r consumes: anObj 
                              produces: anotherObj 
                              machine: self) 
                       then: true asGoal ].
    ^ g
\end{minted}
\begin{minted}[fontsize=\footnotesize]{smalltalk}
testFirstMachine
    | g |

    "McCulloch's first machine"
    g := [ :a | self mcculloch proves: a relates: a ] fresh.
    self assert: (g solutions atMost: 1) equals: {#(3 2 3) asCons}.

    "Montecarlo lock"
    g := [ :a | self mclock proves: a relates: a ] fresh.
    self
        assert: ((g solutions atMost: 1) collect: #asLinkedList)
        equals: {#(5 4 6 4 2 5 4 6 4 2)}
\end{minted}
\end{frame}

\end{document}

